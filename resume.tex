% Copyright 2013 Christophe-Marie Duquesne <chmd@chmd.fr>
% Copyright 2014 Mark Szepieniec <http://github.com/mszep>
% 
% ConText style for making a resume with pandoc. Inspired by moderncv.
% 
% This CSS document is delivered to you under the CC BY-SA 3.0 License.
% https://creativecommons.org/licenses/by-sa/3.0/deed.en_US

\startmode[*mkii]
  \enableregime[utf-8]  
  \setupcolors[state=start]
\stopmode

\setupcolor[hex]
\definecolor[titlegrey][h=757575]
\definecolor[sectioncolor][h=397249]
\definecolor[rulecolor][h=9cb770]

% Enable hyperlinks
\setupinteraction[state=start, color=sectioncolor]

\setuppapersize [A4][A4]
\setuplayout    [width=middle, height=middle,
                 backspace=20mm, cutspace=0mm,
                 topspace=10mm, bottomspace=20mm,
                 header=0mm, footer=0mm]

%\setuppagenumbering[location={footer,center}]

\setupbodyfont[11pt, helvetica]

\setupwhitespace[medium]

\setupblackrules[width=31mm, color=rulecolor]

\setuphead[chapter]      [style=\tfd]
\setuphead[section]      [style=\tfd\bf, color=titlegrey, align=middle]
\setuphead[subsection]   [style=\tfb\bf, color=sectioncolor, align=right,
                          before={\leavevmode\blackrule\hspace}]
\setuphead[subsubsection][style=\bf]

\setuphead[chapter, section, subsection, subsubsection][number=no]

%\setupdescriptions[width=10mm]

\definedescription
  [description]
  [headstyle=bold, style=normal,
   location=hanging, width=18mm, distance=14mm, margin=0cm]

\setupitemize[autointro, packed]    % prevent orphan list intro
\setupitemize[indentnext=no]

\setupfloat[figure][default={here,nonumber}]
\setupfloat[table][default={here,nonumber}]

\setuptables[textwidth=max, HL=none]

\setupthinrules[width=15em] % width of horizontal rules

\setupdelimitedtext
  [blockquote]
  [before={\setupalign[middle]},
   indentnext=no,
  ]


\starttext

\subsection[vivien-giraud]{Vivien Giraud}

\placetable[none]{}
\starttable[|l|r|]
\HL
\NC 12 avenue de la cible
\NC jobs@viviengiraud.fr
\NC\AR
\NC 13100 Aix-en-Provence
\NC https://fr.linkedin.com/in/viviengiraud/en
\NC\AR
\NC France
\NC +33 (0)6 21.43.30.87
\NC\AR
\HL
\stoptable

\section[embedded-softwate-developper]{EMBEDDED SOFTWATE DEVELOPPER}

\startblockquote
Passionate about hacking and Do It Yourself philosophy, I enjoy reverse
engineering all I have on hand and developing tool to automate my
research.
\stopblockquote

\subsubsection[education]{Education}

\startdescription{2011-2014}
  {\bf Bachelor, Real Time and Embedded Systems}; Ingésup
  (Aix-en-Provence)

  {\em Project Manager for TVOS and Litchi Project and in charge of
  laboratory}
\stopdescription

\subsubsection[experience]{Experience}

\startdescription{{\bf INSIDE Secure} {\em (2014-)}}
  Pre-sales and support of MatrixSSE and MatrixHCE sofware

  {\bf \useURL[url1][http://www.insidesecure.com/Markets-solutions/Payment-and-Mobile-Banking/MatrixSSE2][][MatrixSSE]\from[url1]}
  Secure Element in Software

  {\bf \useURL[url2][http://www.insidesecure.com/Products-Technologies/Mobile-Payment-and-Banking/Matrix-HCE][][MatrixHCE]\from[url2]}
  HCE based Mobile Payment Application Platform
\stopdescription

\startdescription{{\bf NEOTION} {\em (2011-2014)}}
  Development of a debugger and a smartcard reader for QEMU

  {\bf NeoGDB} Based on GDB and OpenOCD

  {\bf SmartCard Reader} I wrote specifications of the proprietary usb
  protocol the company would use then search for appropriate
  electronical components, read datasheets then Hardware design the
  smartcard reader. Based on AVR32 chip, I wrote code so as to make the
  bootloader, the USB communication and firmware update.
\stopdescription

\subsubsection[technical-experience]{Technical Experience}

\startdescription{{\bf TVOS:}}
  Development of an open-source TV based on Linux and Litchi Project. I
  was in charge of low level development and packaging firmware image.
\stopdescription

\startdescription{{\bf Litchi Project:}}
  Hardware and software development of a low cost video game console
  allowing you to use it as a media center.
\stopdescription

\startdescription{{\bf Multi-voltage JTAG/Serial dongle:}}
  Hardware design of a JTAG/Serial dongle able to works on different
  voltage (1,8v - 3,3v - 5v) or can use any voltage between 1,8v to
  5,0v. Based on
  \useURL[url3][http://www.ftdichip.com/Products/ICs/FT4232H.htm][][FT4232H]\from[url3]
  chip you can, at the same time, control your CPU using JTAG and
  received debug information via Serial port.
\stopdescription

\startdescription{{\bf Brutas:}}
  I wrote tool to automate my research in security flaws on embedded
  products. Thanks to it I was able to find various breaches in
  surveillance cameras.

  Knowledge in {\bf LibUSB}, {\bf LibFTDI}, {\bf ARM Assembly},
  {\bf AVR32}
\stopdescription

\subsubsection[extra-section]{Extra Section}

\startitemize
\item
  Languages:

  \startitemize[packed]
  \item
    French (native speaker)
  \item
    English (TOEIC 760)
  \item
    Chinese, Japanese and Hungarian concepts
  \stopitemize
\item
  Roller hockey, Guitar player, Reverse engineering
\stopitemize

\stoptext
